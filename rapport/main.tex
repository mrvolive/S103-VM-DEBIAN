\documentclass[a4paper, 12pt]{report}

%====================== PACKAGES ======================

\usepackage[french]{babel}
\usepackage[utf8x]{inputenc}
%pour gérer les positionnement d'images
\usepackage{float}
\usepackage{amsmath}
\usepackage{graphicx}
\usepackage[colorinlistoftodos]{todonotes}
\usepackage{url}
%pour les informations sur un document compilé en PDF et les liens externes / internes
\usepackage{hyperref}
%pour la mise en page des tableaux
\usepackage{array}
\usepackage{tabularx}
%pour utiliser \floatbarrier
%\usepackage{placeins}
%\usepackage{floatrow}
%espacement entre les lignes
\usepackage{setspace}
%modifier la mise en page de l'abstract
\usepackage{abstract}
%police et mise en page (marges) du document
\usepackage[T1]{fontenc}
\usepackage[top=2cm, bottom=2cm, left=2cm, right=2cm]{geometry}
%Pour les galerie d'images
\usepackage{subfig}

%====================== INFORMATION ET REGLES ======================

%rajouter les numérotation pour les \paragraphe et \subparagraphe
\setcounter{secnumdepth}{4}
\setcounter{tocdepth}{4}

\hypersetup{							% Information sur le document
pdfauthor = {Sugdenaz EKICI,
			Yahia KHERZA,
			Olivier MARAVAL,
    		Valentin VIRET-JACQUOT},			% Auteurs
pdftitle = {1_Etude},			% Titre du document
pdfsubject = {Mémoire de Projet},		% Sujet
pdfkeywords = {Tag1, Tag2, Tag3, ...},	% Mots-clefs
pdfstartview={FitH}}					% ajuste la page à la largueur de l'écran
%pdfcreator = {MikTeX},% Logiciel qui a crée le document
%pdfproducer = {}} % Société avec produit le logiciel

%======================== DEBUT DU DOCUMENT ========================

\begin{document}

\addtocontents{toc}{\protect\thispagestyle{empty}}

%régler l'espacement entre les lignes
\newcommand{\HRule}{\rule{\linewidth}{0.5mm}}

%page de garde
\begin{titlepage}
\begin{center}

% Upper part of the page. The '~' is needed because only works if a paragraph has started.
\includegraphics[width=0.5\textwidth]{./images/InfoLogoQuadriH.png}~\\[1cm]

\textsc{\LARGE SAE 1.03 - BUT INFORMATIQUE - GROUPE x}\\[1.5cm]

\textsc{\Large }\\[0.5cm]

% Title
\HRule \\[0.4cm]

{\huge \bfseries Dossier d'étude et de choix d'un poste de travail pour le développement\\[0.4cm] }

\HRule \\[1.5cm]

% Author and supervisor
\begin{minipage}{0.4\textwidth}
\begin{flushleft} \large
\emph{Auteur:}\\
Sugdenaz \textsc{Ekici}(\textit{A1})\\
Yahia \textsc{Kherza}(\textit{A1})\\
Olivier \textsc{Maraval}(\textit{A1})\\
Valentin \textsc{Viret-Jacquot}(\textit{A1})
\end{flushleft}
\end{minipage}
\begin{minipage}{0.4\textwidth}
\begin{flushright} \large
\emph{Client:} \\
Karine \textsc{Deschinkel}\\
\emph{Référent:} \\
Olivier \textsc{Maraval}
\end{flushright}
\end{minipage}

\vfill

% Bottom of the page
{\large \today}

\end{center}
\end{titlepage}

%page blanche
\newpage
~
%ne pas numéroter cette page
\thispagestyle{empty}
\newpage

\input{./abstract.tex}


\tableofcontents
\thispagestyle{empty}

%ne pas numéroter le sommaire

\newpage

%espacement entre les lignes d'un tableau
\renewcommand{\arraystretch}{1.5}

%====================== INCLUSION DES PARTIES ======================

~
\thispagestyle{empty}


\newpage
%recommencer la numérotation des pages à "7"
\setcounter{page}{7}

\chapter{Présentation du projet}

Intro\footnotemark\\
%note en bas de page

\section{Sujet}

Bla(cf. fig. 1.1)\\

%inclusion d'une mage dans le document
\begin{figure}[!h]
\begin{center}
%taille de l'image en largeur
%remplacer "width" par "height" pour régler la hauteur
\includegraphics[width=15cm]{presentation/schema}
\end{center}
%légende de l'image
\caption{Schéma descriptif}
\end{figure}

%Contenu de la note précédemment marquée avec \footnotemark
\footnotetext{Note bas de page "intro"}

Bla
%retour à la ligne (alinea)

Bla\\
%saut de paragraphe

Bla

\newpage

\section{Problématique soulevée}

Bla

\begin{center}
Problématique du sujet
\end{center}

\section{Hypothèse de solution}

%Quoi :
Bla\\

Voici une liste :
\begin{itemize}
\item item 1;
\item item 2;
\item item 3;
\item item 4.
\end{itemize}

Bla\\

%Comment :
Bla

Bla\footnotemark\\

%Detail :
Bla(cf. ref. \cite{cite6}).
%citation référencé dans le document "bibliographie.bib" inclus à la fin du document

\footnotetext{Note bas de page "bla"}

\input{./existant.tex}
 
\chapter{Votre demande}

\section{Résumé de votre demande}

Votre requête concerne une configuration de travail personnalisée pour le développement d'applications web. 

Vous nous avez spécifié un certain nombre d'outils et d'environnements de développement que vous souhaitez avoir à disposition, notamment : 
\begin{itemize}
\item MariaDB / MySQL pour les bases de données, 
\item Python3 avec la possibilité d'avoir des environnements virtuels, Flask et un serveur web.
\end{itemize}

\section{Notre méthode}

Au vue de votre demande plutôt ouverte, notre équipe vous a détaillé différentes possibilités dans la suite de ce document et vous a guidé vers l’option qui lui semble la plus adaptée.

\subsection{Concernant l'architecture matérielle}

Nous vous avons proposé quatre architectures matérielles possibles : deux pour un poste de travail portable et deux pour un poste de travail fixe. Vous avez à chaque fois une option économique et une option confort.

Nous avons inclus dans nos propositions les périphériques d'usage habituels tels qu'un clavier, une souris, un écran, etc. De même pour les ordinateurs portables afin de pouvoir les utiliser comme poste fixe.

Le prix de chaque composant est indiqué. Nous avons intégré un lien web vers la description de chaque composant à toute fins utiles. Vous trouverez également un coût prévisionnel pour chaque configuration.

\subsection{Concernant l'architecture logicielle}

\subsubsection{Le système d'exploitation}

Pour des raisons de sécurité, de flexibilité et de compatibilité, notre équipe s’est arrêtée sur le système d’exploitation Linux. 

Nous vous avons proposé différentes distributions et environnements de bureau afin que vous puissiez rapidement prendre en main votre nouveau poste de travail.

\subsubsection{La suite logicielle}

Notre équipe a décidé de vous proposer une suite logicielle complète vous permettant de commencer immédiatement le développement d’applications web.

Chacun des logiciels que nous avons sélectionné s’accompagne d’une courte description vous indiquant son utilité.

Pour chaque logiciel payant de cette liste nous vous avons indiqué une alternative gratuite en vous indiquant les inconvénients potentiels à un tel choix. Dans le cas où les fonctionnalités perdues s'avéraient anecdotiques, nous avons uniquement inclus l’alternative gratuite.


\chapter{Autre partie}

Dans cette partie nous cherchons à décrire dans un premier temps [...], puis, c[...].

\section{Partie 1}

Intro

\subsection{Sous-partie 1}

\begin{figure}[!ht]
\begin{center}
\includegraphics[height=12cm]{autre_partie/image1}
\end{center}
\caption[autre partie générale]{autre partie image 1\protect\footnotemark}
%\floatfoot{Source: (Citation command)}
% avec le package "floatrow"
\end{figure}

%footnote protected pour apparaitre dans la légende d'une image
\footnotetext{Schéma d'après : \textit{Auteur 1 \& Propriétaire image}, LICENCE (cf. ref. \cite{cite4})}

\newpage{}

\subsection{Sous-partie 2}

\begin{figure}[!ht]
\begin{center}
\includegraphics[height=12cm]{autre_partie/image2}
\end{center}
\caption[autre partie]{autre partie globale de notre quelque chose}
\end{figure}

Nous retrouvons ici, blabla\footnote{Application bla - Interface blabla} [...].

\subsubsection{Sous-sous-partie 1}

Le bla (cf. ref. \cite{cite6}) est [...]:

\begin{itemize}
\item item1;
\item item2;
\item item3;
\item item4;
\item item5.
\end{itemize}

\newpage

\subsubsection{Sous-sous-partie 2}

%Les lignes :
% \setcounter{secnumdepth}{4}
% \setcounter{tocdepth}{4}
%dans le fichier "main.tex" permettent de faire en sorte que les paragraphes soient interprété comme des titres de niveau 5
\paragraph{Paragraphe 1 (agissant comme titre niveau 5)}
%forcer un saut de ligne
~\\
\hskip7mm

\begin{figure}[!ht]
\begin{center}
\includegraphics[height=6cm]{autre_partie/image3}
\end{center}
\caption[Structure d'unz autre chose]{Structure d'une autre chose\protect\footnotemark}
\end{figure}

Ce schéma représente bla.

\footnotetext{Schéma et explication d'après le wiki bla (cf. ref. \cite{cite0})}

\paragraph{Paragraphe 2}
~\\
\hskip7mm

%fixer les floats précédemment définis
%\FloatBarrier

Bla

\subparagraph{Sous-paragraphe 1}
~\\
\hskip7mm

Bla

\begin{figure}[H]
\begin{center}
\includegraphics[height=10cm]{autre_partie/image4}
\end{center}
\caption{Diagramme de truc}
\end{figure}

\subparagraph{Sous-paragraphe 2}
~\\
\hskip7mm

Bla\\

Bla

\subparagraph{Sous-paragraphe 3}
~\\
\hskip7mm

Bla

\subsubsection{Sous-sous-partie 3}

Bla

\section{Partie 2}

Bla

\footnotetext{D'après le schéma disponible sur la documentfation officielle disponible sur le site blalbla}

Bla

\subsection{Sous-partie 1}

Bla

\subsection{Sous-partie 2}

Bla

\paragraph*{Paragraphe 1 (n'apparaitra pas dans l'index)}
Bla

\paragraph*{Paragraphe 2}
Bla

\paragraph*{Paragraphe 3}
Bla

\subsection{Sous-partie 3}

Bla

\input{./resultats.tex}

\input{./bilan.tex}

\appendixpage

\begin{figure}
	\centering
	\includegraphics[width=0.7\linewidth]{images/mate-terminal}
	\caption{}
	\label{fig:Capture d'écran de l'environnement de Bureau Mate}
\end{figure}

\newpage

%récupérer les citation avec "/footnotemark"
\nocite{*}

%choix du style de la biblio
\bibliographystyle{plain}
%inclusion de la biblio
\bibliography{bibliographie.bib}
%voir wiki pour plus d'information sur la syntaxe des entrées d'une bibliographie

\end{document}