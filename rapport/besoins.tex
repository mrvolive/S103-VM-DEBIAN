\chapter{Votre demande}

\section{Résumé de votre demande}

Votre requête concerne une configuration de travail personnalisée pour le développement d'applications web. 

Vous nous avez spécifié un certain nombre d'outils et d'environnements de développement que vous souhaitez avoir à disposition, notamment : 
\begin{itemize}
\item MariaDB / MySQL pour les bases de données, 
\item Python3 avec la possibilité d'avoir des environnements virtuels, Flask et un serveur web.
\end{itemize}

\section{Notre méthode}

Au vue de votre demande plutôt ouverte, notre équipe vous a détaillé différentes possibilités dans la suite de ce document et vous a guidé vers l’option qui lui semble la plus adaptée.

\subsection{Concernant l'architecture matérielle}

Nous vous avons proposé quatre architectures matérielles possibles : deux pour un poste de travail portable et deux pour un poste de travail fixe. Vous avez à chaque fois une option économique et une option confort.

Nous avons inclus dans nos propositions les périphériques d'usage habituels tels qu'un clavier, une souris, un écran, etc. De même pour les ordinateurs portables afin de pouvoir les utiliser comme poste fixe.

Le prix de chaque composant est indiqué. Nous avons intégré un lien web vers la description de chaque composant à toute fins utiles. Vous trouverez également un coût prévisionnel pour chaque configuration.

\subsection{Concernant l'architecture logicielle}

\subsubsection{Le système d'exploitation}

Pour des raisons de sécurité, de flexibilité et de compatibilité, notre équipe s’est arrêtée sur le système d’exploitation Linux. 

Nous vous avons proposé différentes distributions et environnements de bureau afin que vous puissiez rapidement prendre en main votre nouveau poste de travail.

\subsubsection{La suite logicielle}

Notre équipe a décidé de vous proposer une suite logicielle complète vous permettant de commencer immédiatement le développement d’applications web.

Chacun des logiciels que nous avons sélectionné s’accompagne d’une courte description vous indiquant son utilité.

Pour chaque logiciel payant de cette liste nous vous avons indiqué une alternative gratuite en vous indiquant les inconvénients potentiels à un tel choix. Dans le cas où les fonctionnalités perdues s'avéraient anecdotiques, nous avons uniquement inclus l’alternative gratuite.
