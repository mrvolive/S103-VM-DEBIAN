\chapter{Autre partie}

Dans cette partie nous cherchons à décrire dans un premier temps [...], puis, c[...].

\section{Partie 1}

Intro

\subsection{Sous-partie 1}

\begin{figure}[!ht]
\begin{center}
\includegraphics[height=12cm]{autre_partie/image1}
\end{center}
\caption[autre partie générale]{autre partie image 1\protect\footnotemark}
%\floatfoot{Source: (Citation command)}
% avec le package "floatrow"
\end{figure}

%footnote protected pour apparaitre dans la légende d'une image
\footnotetext{Schéma d'après : \textit{Auteur 1 \& Propriétaire image}, LICENCE (cf. ref. \cite{cite4})}

\newpage{}

\subsection{Sous-partie 2}

\begin{figure}[!ht]
\begin{center}
\includegraphics[height=12cm]{autre_partie/image2}
\end{center}
\caption[autre partie]{autre partie globale de notre quelque chose}
\end{figure}

Nous retrouvons ici, blabla\footnote{Application bla - Interface blabla} [...].

\subsubsection{Sous-sous-partie 1}

Le bla (cf. ref. \cite{cite6}) est [...]:

\begin{itemize}
\item item1;
\item item2;
\item item3;
\item item4;
\item item5.
\end{itemize}

\newpage

\subsubsection{Sous-sous-partie 2}

%Les lignes :
% \setcounter{secnumdepth}{4}
% \setcounter{tocdepth}{4}
%dans le fichier "main.tex" permettent de faire en sorte que les paragraphes soient interprété comme des titres de niveau 5
\paragraph{Paragraphe 1 (agissant comme titre niveau 5)}
%forcer un saut de ligne
~\\
\hskip7mm

\begin{figure}[!ht]
\begin{center}
\includegraphics[height=6cm]{autre_partie/image3}
\end{center}
\caption[Structure d'unz autre chose]{Structure d'une autre chose\protect\footnotemark}
\end{figure}

Ce schéma représente bla.

\footnotetext{Schéma et explication d'après le wiki bla (cf. ref. \cite{cite0})}

\paragraph{Paragraphe 2}
~\\
\hskip7mm

%fixer les floats précédemment définis
%\FloatBarrier

Bla

\subparagraph{Sous-paragraphe 1}
~\\
\hskip7mm

Bla

\begin{figure}[H]
\begin{center}
\includegraphics[height=10cm]{autre_partie/image4}
\end{center}
\caption{Diagramme de truc}
\end{figure}

\subparagraph{Sous-paragraphe 2}
~\\
\hskip7mm

Bla\\

Bla

\subparagraph{Sous-paragraphe 3}
~\\
\hskip7mm

Bla

\subsubsection{Sous-sous-partie 3}

Bla

\section{Partie 2}

Bla

\footnotetext{D'après le schéma disponible sur la documentfation officielle disponible sur le site blalbla}

Bla

\subsection{Sous-partie 1}

Bla

\subsection{Sous-partie 2}

Bla

\paragraph*{Paragraphe 1 (n'apparaitra pas dans l'index)}
Bla

\paragraph*{Paragraphe 2}
Bla

\paragraph*{Paragraphe 3}
Bla

\subsection{Sous-partie 3}

Bla