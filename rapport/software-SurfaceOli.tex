\chapter{Proposition d'architecture logicielle}

\section{Système d'exploitation}

Pour des raisons de sécurité, de flexibilité et de compatibilité, notre équipe s’est arrêtée sur le système d’exploitation Linux. \\

Nous vous avons proposé différentes distributions et environnements de bureau afin que vous puissiez rapidement prendre en main votre nouveau poste de travail.


\subsection{Distribution}

\begin{table}[ht]
	\centering
	\newcolumntype{C}{>{\centering\arraybackslash}p{3cm}}
	\begin{tabularx}{\textwidth}{|C|X|X|}
		\hline
		\textbf{Distribution} & \textbf{Avantages} & \textbf{Inconvénient} \\
		\hline
		\textbf{Ubuntu} & Dernières mises à jour des logiciels ; inclus dans ses dépots une sélection de logiciel et drivers propriétaire & Peut-être moins stable\\
		\hline
		\textbf{Debian} & Globalement plus stable & Installer un logiciel propriétaire peut s'avérer plus compliqué ; peut impacter les performances de certains composants \\
		\hline
	\end{tabularx}
	\caption{Avantages et inconvénients des distributions}
	\label{tab:exemple}
\end{table}

\subsection{Environnement de Bureau}

\begin{table}[ht]
	\centering
	\newcolumntype{C}{>{\centering\arraybackslash}p{3cm}}
	\begin{tabularx}{\textwidth}{|C|X|X|}
		\hline
		\textbf{Distribution} & \textbf{Avantages} & \textbf{Capture d'écran} \\
		\hline
		\textbf{Mate} & L'avantage principal de Mate est sa légèreté, son interface claire et minimaliste permet d'exploiter au maximum les performances de votre machine. & \\
		\hline
		\textbf{KDE} & Globalement plus stable & \\
		\hline
	\end{tabularx}
	\caption{Avantages et inconvénients des distributions}
	\label{tab:exemple}
\end{table}


Organisation du disque; comptes utilisateurs; etc.

\section{Logiciels}

bureau / environnement si plusieurs choix possibles; éditeur; outil de développement, BDD; serveur web; etc.\\

Si les logiciels sont payants les budgéter.
