\chapter{Configuration en mode texte}

\section{Pré-requis}
Connectez vous avec le compte root en utilisant les informations suivantes :
\begin{itemize}
	\item \textbf{login :} \texttt{root}
	\item \textbf{Password :} \texttt{\$sae3*2023\$}
\end{itemize}

\section{Ajouter les dépôts supplémentaires}
Ouvrez le fichier \textbf{sources.list} à l'aide d'un éditeur de texte. Par exemple, vous pouvez utiliser la commande suivante pour ouvrir le fichier avec l'éditeur de texte \texttt{nano} :
\begin{lstlisting}
	nano /etc/apt/sources.list
\end{lstlisting}

Dans le fichier \textbf{sources.list}, vous verrez des lignes commençant par \textbf{"deb"} suivies d'une URL. Ces lignes représentent les dépôts principaux (main). Pour ajouter les dépôts \textbf{"contrib"} et \textbf{"non-free"}, vous devez les ajouter à la fin de chaque ligne correspondante.

Vous pouvez aussi commenter la première ligne en y ajoutant \texttt{\#} ou la supprimer si vous ne souhaitez pas utiliser les paquets du DVD (qui sont inutiles si vous avez une connexion internet)\\

Votre fichier devrait avoir les lignes suivantes
\begin{lstlisting}[style=tf]
	deb http://ftp.fr.debian.org/debian bookworm main non-free-firmware contrib non-free
	deb-src http://ftp.fr.debian.org/debian bookworm non-free-firmware main contrib non-free
	
	deb http://ftp.fr.debian.org/debian-security/ bookworm-security main non-free-firmware contrib non-free
	deb-src http://ftp.fr.debian.org/debian-security/ bookworm-security main non-free-firmware contrib non-free
	
	deb http://ftp.fr.debian.org/debian bookworm-updates main non-free-firmware contrib non-free
	deb-src http://ftp.fr.debian.org/debian bookworm-updates main non-free-firmware contrib non-free
\end{lstlisting}

Enfin, avec \texttt{nano}, vous pouvez valider vos modification en appuyant sur les touches \textbf{CTRL + O} puis \textbf{Entrée} et enfin \textbf{CTRL+X}

Vous pouvez maintenant mettre à jour la liste de paquets en utilisant la commande suivante :

\begin{lstlisting}
	apt update
\end{lstlisting}

\section{Installer le support souris en mode texte}

Utilisez la commande suivante pour installer le paquet \texttt{gpm}, qui fournit le support de la souris dans le shell :

\begin{lstlisting}
	apt install -y gpm
\end{lstlisting}

Maintenant, vous pouvez utiliser \texttt{gpm} pour sélectionner, copier et coller du texte dans une console. Voici quelques commandes utiles :
\begin{itemize}
\item \textbf{Pour sélectionner du texte :} Maintenez le \textbf{bouton gauche} de la souris enfoncé et faites glisser la souris sur le texte que vous souhaitez sélectionner.
\item \textbf{Pour copier et coller le texte sélectionné :} Appuyez sur le \textbf{bouton du milieu} pour copier le texte sélectionné dans le presse-papiers.
\item \textbf{Pour étendre la sélection :} Appuyez sur le \textbf{bouton droit} de la souris pour sélectionner le texte jusqu'au prochain espace. Vous pouvez aussi effectuer un \textbf{double-clic} pour le même effet.
\item \textbf{Pour sélectionner la ligne entière :} Effectuez un \textbf{triple-clic} pour sélectionner l'entièreté d'une ligne.
\end{itemize}

\section{Modification du groupe staff}

Exécutez la commande suivante pour modifier le groupe \texttt{staff} et lui donner le \texttt{gid} 500

\begin{lstlisting}
	groupmod -g 500 staff
\end{lstlisting}

\section{Modification de l'utilisateur msalomon}

Executez les commandes suivantes pour assigner l'utilisateur aux groupes \texttt{staff} (qui sera son groupe principal) et au groupe \texttt{adm} puis enregistrer les informations suivantes :
\begin{itemize}
	\item \textbf{Full name} = Michel Salomon
	\item \textbf{Room} = E004
	\item \textbf{Work phone} = 03.84.58.77.76
\end{itemize}
La dernière commande supprimera le groupe au nom de l'utilisateur.

\begin{lstlisting}
usermod -u 2500 -g 500 -G adm msalomon
chfn -f "Michel Salomon" -r "E004" -w "03.84.58.77.76" msalomon
groupedel msalomon
\end{lstlisting}
