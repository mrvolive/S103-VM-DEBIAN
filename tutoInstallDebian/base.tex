\chapter{Installation du Système de Base}

\section{Accéder au programme d'installation}

Si votre pc démarre directement sur le programme d'installation, vous pouvez vous rendre directement à la section suivante.

\begin{itemize}
\item Insérez l'ISO Débian suivant : \textit{\href{https://cdimage.debian.org/debian-cd/current/amd64/iso-dvd/debian-12.2.0-amd64-DVD-1.iso}{debian-12.2.0-amd64-DVD-1.iso}}
\item Démarrez votre ordinateur et accédez au menu de démarrage du BIOS en appuyant sur la touche appropriée (généralement F2, F10, F12 ou Suppr) au démarrage de l'ordinateur.
\item Sélectionnez la clé USB comme périphérique de démarrage et appuyez sur Entrée pour démarrer votre ordinateur à partir de la clé USB.\\
\end{itemize}

Vous vous déplacerez dans le menu d'installation avec les \textbf{Touches Fléchées}. La touche \textbf{Espace} sélectionne une option et la touche \textbf{Entrée} valide un choix.\\

Vous serez accueilli par un écran d'installation vous indiquant plusieurs options.

\section{Ecran d'installation}
\begin{itemize}
	\item Sélectionnez la deuxième option : \textbf{Install}
\end{itemize}

\section{Select a language}
\begin{itemize}
	\item Choisissez \textbf{French - Français} dans la liste et appuyez sur \textbf{Entrée}
\end{itemize}

\section{Choix de votre situation géographique}
\begin{itemize}
	\item Choisissez \textbf{France} dans la liste et appuyez sur \textbf{Entrée}
\end{itemize}

\section{Configurer le clavier}
\begin{itemize}
	\item Choisissez \textbf{Français} dans la liste
\end{itemize}

\section{Configurer le réseau}

\begin{itemize}
	\item Dans le champ \textbf{Nom de machine} inscrivez le texte : \texttt{poste-dev-1}
	\item Laissez le champ \textbf{Domaine} vide
\end{itemize}

\section{Créer les utilisateurs et choisir les mots de passe}
\begin{itemize}
	\item Dans le champ \textbf{Mot de passe du superutilisateur (" root ")} écrivez : \texttt{\$sae3*2023\$}
	\item Réécrivez le même mot de passe dans le champ \textbf{Confirmation du mot de passe}
	\item Dans le champ \textbf{Nom complet du nouvel utilisateur} entrez : \texttt{Michel Salomon}
	\item Dans le champ \textbf{Identifiant pour le compte utilisateur} entrez : \texttt{msalomon}
	\item Dans le champ \textbf{Mot de passe pour le nouvel utilisateur} entrez : \texttt{CqriT}
	\item Réécrivez le même mot de passe dans le champ \textbf{Confirmation du mot de passe}
\end{itemize}

\section{Partitionner les disques}
\begin{itemize}
	\item Dans la liste \textbf{Méthode de partionnement} sélectionnez la dernière option : \textbf{Manuel}
	\item Sélectionnez dans la liste le disque sur lequel vous souhaitez installer Debian.
	\item Répondez \textbf{Oui} à la question : \textit{"Faut-il créer une nouvelle table des partitions sur ce disque ?"}
\end{itemize}

\section{Partition /}
	\begin{itemize}
	\item Sélectionnez la nouvelle option venant d'apparaître sur la liste et terminant par \textbf{"Espace libre"}
		\begin{itemize}
			\item Sélectionnez \textbf{Créer une nouvelle partition}
			\item Dans le champ \textbf{Nouvelle taille de la partition} entrez la valeur : \texttt{10 GB}
			\item Dans la liste \textbf{Type de la nouvelle partition} sélectionnez : \textbf{Primaire}
			\item Dans la liste \textbf{Emplacement de la nouvelle partition} sélectionnez : \textbf{Début}
			\item Dans la liste \textbf{Caractéristiques de la partition} sélectionnez : \textbf{Fin du paramétrage de cette partition}\\	
		\end{itemize}
	\end{itemize}

La nouvelle partition a dû apparaitre en n°1 dans la liste.
		
\subsection{Partition swap}
	\begin{itemize}
		\item Sélectionnez de nouveau l'option terminant par \textbf{"Espace libre"}
		\begin{itemize}
			\item Sélectionnez \textbf{Créer une nouvelle partition}
			\item Dans le champ \textbf{Nouvelle taille de la partition} entrez la valeur : \texttt{2 GB}
			\item Dans la liste \textbf{Type de la nouvelle partition} sélectionnez : \textbf{Logique}
			\item Dans la liste \textbf{Emplacement de la nouvelle partition} sélectionnez : \textbf{Début}
			\item Dans la liste \textbf{Caractéristiques de la partition} sélectionnez : \textbf{Utiliser comme} et choisissez : \textbf{espace d'échange ("swap")} dans la liste qui vient d'apparaitre.
			\item De retour dans la liste \textbf{Caractéristiques de la partition} sélectionnez : \textbf{Fin du paramétrage de cette partition}
		\end{itemize}	
	\end{itemize}
	
\subsection{Partition /opt}
\begin{itemize}
	\item Sélectionnez de nouveau l'option terminant par \textbf{"Espace libre"}
	\begin{itemize}
		\item Sélectionnez \textbf{Créer une nouvelle partition}
		\item Dans le champ \textbf{Nouvelle taille de la partition} entrez la valeur : \texttt{10 GB}
		\item Dans la liste \textbf{Type de la nouvelle partition} sélectionnez : \textbf{Logique}
		\item Dans la liste \textbf{Emplacement de la nouvelle partition} sélectionnez : \textbf{Début}
		\item Dans la liste \textbf{Caractéristiques de la partition} sélectionnez : \textbf{Point de montage} et choisissez : \textbf{/opt} dans la liste qui vient d'apparaitre.
		\item De retour dans la liste \textbf{Caractéristiques de la partition} sélectionnez : \textbf{Fin du paramétrage de cette partition}
	\end{itemize}	
\end{itemize}

\subsection{Partition /home}
\begin{itemize}
	\item Sélectionnez de nouveau l'option terminant par \textbf{"Espace libre"}
	\begin{itemize}
		\item Sélectionnez \textbf{Créer une nouvelle partition}
		\item Dans le champ \textbf{Nouvelle taille de la partition} entrez la valeur : \texttt{100\%}
		\item Dans la liste \textbf{Type de la nouvelle partition} sélectionnez : \textbf{Logique}
		\item Dans la liste \textbf{Emplacement de la nouvelle partition} sélectionnez : \textbf{Début}
		\item Dans la liste \textbf{Caractéristiques de la partition} sélectionnez : \textbf{Point de montage} et choisissez : \textbf{/home} dans la liste qui vient d'apparaitre.
		\item De retour dans la liste \textbf{Caractéristiques de la partition} sélectionnez : \textbf{Fin du paramétrage de cette partition}
	\end{itemize}	
\end{itemize}

\subsection{Finalisation des partitions}
\begin{itemize}
	\item Sélectionnez l'option \textbf{"Terminer le partitionnement et appliquer les changements"}
	\item Répondez \textbf{Oui} à la question : \textit{Faut-il appliquer les changements sur les disques ?}\\
\end{itemize}

L'installation du système de base va alors commencer, vous pourrez suivre son avancement grâce à la barre de chargement. Ce processus durera plus ou moins longtemps selon les caractéristiques de votre machine.

\section{Configurer l'outil de gestion des paquets}
\begin{itemize}
	\item Répondre \textbf{Non} à la question : \textit{Faut-il analyser d'autres supports d'installation ?}
	\item Répondre \textbf{Oui} à la question : \textit{Faut-il utiliser un miroir sur le réseau ?}
	\item Dans la liste \textbf{Pays du miroir de l'archive Debian} sélectionnez : \textbf{France}
	\item Dans la liste \textbf{Miroir de l'archive Debian} sélectionnez : \textbf{ftp.fr.debian.org}
	\item Laissez le champ \textbf{Mandataire HTTP} vide
	\item Répondre \textbf{Non} à la question : \textit{Souhaitez-vous participer à l'étude statistique sur l'utilisation des paquets ?}
\end{itemize}

\section{Sélection des logiciels}
\begin{itemize}
	\item Décochez les logiciels sélectionnés par défaut dans la liste \textbf{Logiciels à installer} en utilisant la \textbf{Barre espace}
	\item De la même façon, cochez les logiciels : \textbf{serveur SSH} et \textbf{utilitaires usuels du système} (seules ces deux cases doivent être cochées)
	\item Une fois la sélection effectuée vous pouvez valider avec \textbf{Entrée}
\end{itemize}

\section{Configuration de grub-pc}
\begin{itemize}
	\item Répondre \textbf{Oui} à la question : \textit{"Installer le programme de démarrage GRUB sur le disque principal}
	\item Dans la liste \textbf{Périphérique où sera installé le programme de démarrage} sélectionnez l'option indiquant le disque que vous venez de partitionner.
\end{itemize}

\section{Terminer l'installation}
Si le message \textbf{Installation terminée} s'affiche vous pouvez sélectionner l'option \textbf{Continuer} qui redémarrera l'ordinateur. Pensez à retirer le support d'installation et vous devriez vous retrouver face à un terminal affichant un curseur clignotant juste après le texte : \texttt{poste-dev-1 login:}