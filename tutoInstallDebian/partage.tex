<<<<<<< HEAD
\chapter{Répertoire partagé}

\section{Récupération des Guest Additions}

Récupérer l'image iso des Guest Additions au niveau de \href{https://download.virtualbox.org/virtualbox/}{https://download.virtualbox.org/virtualbox/}, vous pouvez utiliser le lien ci-dessous en remplaçant le mot \texttt{VERSION} par votre numéro de version

\begin{lstlisting}
	https://download.virtualbox.org/virtualbox/VERSION/VBoxGuestAdditions_VERSION.iso
\end{lstlisting}

Démarrer la machine virtuelle et se connecter en super-utilisateur, puis faire les manipulations qui suivent pour insérer l'ISO dans le lecteur virtuel de la machine virtuelle :
\begin{itemize}
	\item Rendez-vous sur l'écran contenant votre VM lancée
	\item Périphériques > Lecteurs Optiques > Choose a disk file...
	\item Sélectionnez l'ISO que vous avez téléchargé.
\end{itemize}

\section{Installation de l'ISO}

Créer le répertoire qui doit servir de point de montage s'il n'existe pas déjà

\begin{lstlisting}
	mkdir /media/cdrom
\end{lstlisting}

Exécutez les commandes suivantes pour monter le CD dans ce répertoire, installer le nécessaire pour l'installation du contenu du disque et enfin lancer l'installateur présent sur ce disque.

\begin{lstlisting}
	mount -t iso9660 /dev/cdrom /media/cdrom
	apt update
	apt install -y build-essential linux-headers-`uname -r`
	cd /media/cdrom
	./VBoxLinuxAdditions.run
\end{lstlisting}

Redémarrer votre machine virtuelle à la fin de l'installation

\section{Vérification}

Reconnectez-vous en super-utilisateur et confirmez qu'il puisse accéder au répertoire partagé en Lecture/Écriture grâce à la commande suivante :

\begin{lstlisting}
	ls -l / |grep localhome|cut -d " " -f 1
\end{lstlisting}

À partir de la deuxième lettre de la chaine de caractère qui apparaît, vous devez avoir, au minimum, rw (normalement rwx).\\

On va maintenant permettre à l'utilisateur msalomon d'avoir les droits en lecture et en écriture sur le dossier en l'ajoutant au groupe vboxsf:

\begin{lstlisting}
	usermod -aG vboxsf msalomon
\end{lstlisting}

=======
\chapter{Répertoire partagé}
>>>>>>> 6e0e4c269c4e06128cc702a8f257ca79eca8d33b
