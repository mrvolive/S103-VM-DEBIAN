\chapter{Configuration réseau}

\section{Adresse IP de la machine virtuelle et de l'hôte}

\subsection{IP de la machine virtuelle}

Vous pouvez accéder à l'adresse IP de la machine virtuelle en utilisant la commande suivante :

\begin{lstlisting}
ip addr show
\end{lstlisting}

Vous devriez avoir deux sections, celle qui vous intéresse est celle qui ne commence pas par \texttt{lo: <LOOPBACK}...

Dans la section pertinente, votre adresse ip se trouve à droite du mot \texttt{inet}\\

exemple : 
\begin{lstlisting}[style=tf]
inet 10.0.2.15/24 brd 10.0.2.255 scope global dynamicenp0s3
\end{lstlisting}
ici, l'ip est \texttt{10.0.2.15}

\subsection{IP de la machine hôte}
Vous pouvez utiliser la commande suivante pour afficher la table de routage :
\begin{lstlisting}
ip route
\end{lstlisting}
Cette dernière vous donnera plusieurs valeurs et il est possible de déduire laquelle correspond à l'adresse ip de la machine hôte :
\begin{lstlisting}[style=tf]
default via 10.0.2.2 dev enp0s3
10.0.2.0/24 dev enp0s3 proto kernel scope link src 10.0.2.15
169.254.0.0/16 dev enp0s3 scope link metric 1000
\end{lstlisting}
ici on peut déduire que l'adresse ip de la machine hôte est \texttt{10.0.2.2}.\\

\section{Fichier de configuration pour SSH}

Passez à l'utilisateur \texttt{msalomon}. Pour cela, utilisez la commande suivante :

\begin{lstlisting}
su - msalomon
\end{lstlisting}

Réduisez l'accès au répertoire \textit{.ssh} en modifiant les droits d'accès avec la commande :

\begin{lstlisting}
chmod 700 ~/.ssh
\end{lstlisting}

Créez un fichier de configuration dans le dossier \texttt{.ssh}, réduisez les droits d'accès au seul propriétaire et ouvrez-le pour modification grâce aux commandes suivantes :
\begin{lstlisting}
touch ~/.ssh/config
chmod 600 ~/.ssh/config
nano .ssh/config
\end{lstlisting}

Écrire les informations suivantes dans le fichier qui vient de s'ouvrir et l'enregistrer

\begin{lstlisting}[style=tf]
Host machineHote
	Hostname 10.0.2.2
	User userHote
	IdentityFile ~/.ssh/id_rsa
\end{lstlisting}

Enfin, vous pourrez copier la clé publique de la paire rsa dans le compte utilisateur sur la vraie machine via la commande suivante :

\begin{lstlisting}
ssh-copy-id -i ~/.ssh/id_rsa.pub machineHote
\end{lstlisting}
